Gadfly 0.4.1
 Code Issues

Gadfly is a system for plotting and visualization based largely on Hadley Wickhams's ggplot2 for R, and Leland Wilkinson's book The Grammar of Graphics.

Getting Started
From the Julia REPL a reasonably up to date version can be installed with

Pkg.add("Gadfly")
This will likely result in half a dozen or so other packages also being installed.

Gadfly is then loaded with.

using Gadfly
Optional: cairo, pango, and fontconfig
Gadfly works best with the C libraries cairo, pango, and fontconfig installed. The PNG, PS, and PDF backends require cairo, but without it the SVG backends (SVG and SVGJS) are still available.

Complex layouts involving text are also somewhat more accurate when pango and fontconfig are available.

Julia's Cairo bindings can be installed with

Pkg.add("Cairo")
Plot invocations
Most interaction with Gadfly is through the plot function. Plots are described by binding data to aesthetics, and specifying a number of plot elements including scales, coordinates, guides, and geometries. Aesthetics are a set of special named variables that are mapped to plot geometry. How this mapping occurs is defined by the plot elements.

This "grammar of graphics" approach tries to avoid arcane incantations and special cases, instead approaching the problem as if one were drawing a wiring diagram: data is connected to aesthetics, which act as input leads, and elements, each self-contained with well-defined inputs and outputs, are connected and combined to produce the desired result.

Plotting arrays
If no plot elements are defined, point geometry is added by default. The point geometry takes as input the x and y aesthetics. So all that's needed to draw a scatterplot is to bind x and y.

MersenneTwister(Base.dSFMT.DSFMT_state(Int32[-870096391,1072918504,-1812426662,1073255081,-733866021,1073404543,807620846,1073368448,1919433844,1072852359  …  -362113007,1073100625,-166402106,1073460158,-1907020342,721295190,-750225566,-1300227565,382,0]),[1.59084,1.7668,1.56624,1.46009,1.79403,1.85415,1.20059,1.29861,1.24684,1.57967  …  1.79452,1.847,1.33149,1.56031,1.65487,1.27811,1.01868,1.70895,1.40328,1.43228],382,UInt32[0x00003039])
# E.g.
plot(x=rand(10), y=rand(10))
x
0.00
0.25
0.50
0.75
1.00
0.0
0.5
1.0
y
Multiple elements can use the same aesthetics to produce different output. Here the point and line geometries act on the same data and their results are layered.

# E.g.
plot(x=rand(10), y=rand(10), Geom.point, Geom.line)
x
0.0
0.5
1.0
0.0
0.2
0.4
0.6
0.8
1.0
y
More complex plots can be produced by combining elements.

# E.g.
plot(x=1:10, y=2.^rand(10),
     Scale.y_sqrt, Geom.point, Geom.smooth,
     Guide.xlabel("Stimulus"), Guide.ylabel("Response"), Guide.title("Dog Training"))
Stimulus
0.0
2.5
5.0
7.5
10.0
1.02
1.12
1.22
1.32
1.42
Response
Dog Training
To generate an image file from a plot, use the draw function. Gadfly supports a number of drawing backends. Each is used similarly.

# define a plot
myplot = plot(..)

# draw on every available backend
draw(SVG("myplot.svg", 4inch, 3inch), myplot)
draw(SVGJS("myplot.svg", 4inch, 3inch), myplot)
draw(PNG("myplot.png", 4inch, 3inch), myplot)
draw(PDF("myplot.pdf", 4inch, 3inch), myplot)
draw(PS("myplot.ps", 4inch, 3inch), myplot)
draw(PGF("myplot.tex", 4inch, 3inch), myplot)
If used from IJulia, the output of plot will be shown automatically.

Plotting data frames
The DataFrames package provides a powerful means of representing and manipulating tabular data. They can be used directly in Gadfly to make more complex plots simpler and easier to generate.

In this form of plot, a data frame is passed to as the first argument, and columns of the data frame are bound to aesthetics by name or index.

# Signature for the plot applied to a data frames.
plot(data::AbstractDataFrame, elements::Element...; mapping...)
The RDatasets package collects example data sets from R packages. We'll use that here to generate some example plots on realistic data sets. An example data set is loaded into a data frame using the dataset function.

using RDatasets
# E.g.
plot(dataset("datasets", "iris"), x="SepalLength", y="SepalWidth", Geom.point)
SepalLength
4
5
6
7
8
2.0
2.5
3.0
3.5
4.0
4.5
SepalWidth
 # E.g.
plot(dataset("car", "SLID"), x="Wages", color="Language", Geom.histogram)
Wages
0
10
20
30
40
50
English
Other
French
Language
0
50
100
150
200
Along with less typing, using data frames to generate plots allows the axis and guide labels to be set automatically.

Functions and Expressions
Along with the standard plot function, Gadfly has some special forms to make plotting functions and expressions more convenient.

plot(f::Function, a, b, elements::Element...)

plot(fs::Array, a, b, elements::Element...)
Some special forms of plot exist for quickly generating 2d plots of functions.

# E.g.
plot([sin, cos], 0, 25)
x
0
5
10
15
20
25
f1
f2
Color
-1.0
-0.5
0.0
0.5
1.0
f(x)
Layers
Gadfly can draw multiple layers to the same plot:

plot(layer(x=rand(10), y=rand(10), Geom.point),
     layer(x=rand(10), y=rand(10), Geom.line))
Or if your data is in a DataFrame:

plot(my_data, layer(x="some_column1", y="some_column2", Geom.point),
              layer(x="some_column3", y="some_column4", Geom.line))
You can also pass different data frames to each layers:

layer(another_dataframe, x="col1", y="col2", Geom.point)
Ordering of layers can be controlled with the order keyword. A higher order number will cause a layer to be drawn on top of any layers with a lower number. If not specified, default order for a layer is 0.

plot(layer(x=rand(10), y=rand(10), Geom.point, order=1),
     layer(x=rand(10), y=rand(10), Geom.line, order=2))
Guide attributes may be added to a multi-layer plots:

plt=plot(layer(x=rand(10), y=rand(10), Geom.point),
         layer(x=rand(10), y=rand(10), Geom.line),
         Guide.XLabel("XLabel"),
         Guide.YLabel("YLabel"),
         Guide.Title("Title"));
Stacking
Plots can also be stacked horizontally with hstack or vertically with vstack. This allows more customization in regards to tick marks, axis labeling, and other plot details than is available with subplot_grid.

p1 = plot(x=[1,2,3], y=[4,5,6])
p2 = plot(x=[1,2,3], y=[6,7,8])
draw(PDF("p1and2.pdf", 6inch, 6inch), vstack(p1,p2))
p3 = plot(x=[5,7,8], y=[8,9,10])
p4 = plot(x=[5,7,8], y=[10,11,12])
draw(PDF("p1to4.pdf", 6inch, 9inch), vstack(hstack(p1,p2),hstack(p3,p4)))
Drawing to backends
Gadfly plots can be rendered to number of formats. Without cairo, or any non-julia libraries, it can produce SVG. Installing cairo gives you access to the PNG, PDF, and PS backends. Rendering to a backend works the same for any of these.

p = plot(x=[1,2,3], y=[4,5,6])
draw(PNG("myplot.png", 12cm, 6cm), p)
Using the SVGJS backend
The SVGJS backend writes SVG with embedded javascript. There are a couple subtlties with using the output from this backend.

Drawing to the backend works like any other.

draw(SVGJS("mammals.js.svg", 6inch, 6inch), p)
If included with an <img> tag, it will display as a static SVG image.

<img src="mammals.js.svg"/>
For the interactive javascript features to be enables, the output either needs to be inluded inline in the HTML page, or include with an object tag, like.

<object data="mammals.js.svg" type="image/svg+xml"></object>
A div element must be placed, and the draw function defined in mammals.js must be passed the id of this element, so it knows where in the document to place the plot.

IJulia
The IJulia project adds Julia support to IPython. This includes a browser based notebook that can inline graphics and plots. Gadfly works out of the box with IJulia, with or without drawing explicity to a backend.

Without a specific call to draw (i.e. just calling plot), the D3 backend is used with a default plot size. The default plot size can be changed with set_default_plot_size.

# E.g.
set_default_plot_size(12cm, 8cm)
Reporting Bugs
This is a new and fairly complex piece of software. Filing an issue to report a bug, counterintuitive behavior, or even to request a feature is extremely valuable in helping me prioritize what to work on, so don't hestitate.


SEARCH
Introduction
Getting Started
Optional: cairo, pango, and fontconfig
Plot invocations
Plotting arrays
Plotting data frames
Functions and Expressions
Layers
Stacking
Drawing to backends
Using the SVGJS backend
IJulia
Reporting Bugs
Themes
Statistic
step
qq
xticks
yticks
x_jitter
y_jitter
binmean
Geometry
boxplot
density
errorbar
bar
histogram
histogram2d
hline
label
line
point
rectbin
smooth
subplot_grid
vline
hexbin
step
ribbon
contour
path
violin
polygon
beeswarm
Guide
xrug
yrug
xlabel
ylabel
xticks
yticks
title
colorkey
annotation
manual_color_key
Scale
x_continuous
y_continuous
x_discrete
y_discrete
color_continuous
color_discrete_hue
color_discrete_manual
color_none
Coord
cartesian
Gadfly Internals
Rendering Pipeline
Last modified by Daniel Jones on 2016-01-20. Generated with Judo.